% Für Bindekorrektur als optionales Argument "BCORfaktormitmaßeinheit", dann
% sieht auch Option "twoside" vernünftig aus
% Näheres zu "scrartcl" bzw. "scrreprt" und "scrbook" siehe KOMA-Skript Doku
\documentclass[12pt,a4paper,titlepage,headinclude,bibtotoc]{scrartcl}


%---- Allgemeine Layout Einstellungen ------------------------------------------

% Für Kopf und Fußzeilen, siehe auch KOMA-Skript Doku
\usepackage[komastyle]{scrpage2}
\pagestyle{scrheadings}
\setheadsepline{0.5pt}[\color{black}]
\automark[section]{chapter}

%Einstellungen für Figuren- und Tabellenbeschriftungen
\setkomafont{captionlabel}{\sffamily\bfseries}
\setcapindent{0em}


%---- Weitere Pakete -----------------------------------------------------------
% Die Pakete sind alle in der TeX Live Distribution enthalten. Wichtige Adressen
% www.ctan.org, www.dante.de

% Sprachunterstützung
\usepackage[ngerman]{babel}

% Benutzung von Umlauten direkt im Text
% entweder "latin1" oder "utf8"
\usepackage[utf8]{inputenc}

% Pakete mit Mathesymbolen und zur Beseitigung von Schwächen der Mathe-Umgebung
\usepackage{latexsym,exscale,stmaryrd,amssymb,amsmath}

% Weitere Symbole
\usepackage[nointegrals]{wasysym}
\usepackage{eurosym}

% Anderes Literaturverzeichnisformat
%\usepackage[square,sort&compress]{natbib}

% Für Farbe
\usepackage{color}

% Zur Graphikausgabe
%Beipiel: \includegraphics[width=\textwidth]{grafik.png}
\usepackage{graphicx}

% Text umfließt Graphiken und Tabellen
% Beispiel:
% \begin{wrapfigure}[Zeilenanzahl]{"l" oder "r"}{breite}
%   \centering
%   \includegraphics[width=...]{grafik}
%   \caption{Beschriftung} 
%   \label{fig:grafik}
% \end{wrapfigure}
\usepackage{wrapfig}

% Mehrere Abbildungen nebeneinander
% Beispiel:
% \begin{figure}[htb]
%   \centering
%   \subfigure[Beschriftung 1\label{fig:label1}]
%   {\includegraphics[width=0.49\textwidth]{grafik1}}
%   \hfill
%   \subfigure[Beschriftung 2\label{fig:label2}]
%   {\includegraphics[width=0.49\textwidth]{grafik2}}
%   \caption{Beschriftung allgemein}
%   \label{fig:label-gesamt}
% \end{figure}
\usepackage{subfigure}

% Caption neben Abbildung
% Beispiel:
% \sidecaptionvpos{figure}{"c" oder "t" oder "b"}
% \begin{SCfigure}[rel. Breite (normalerweise = 1)][hbt]
%   \centering
%   \includegraphics[width=0.5\textwidth]{grafik.png}
%   \caption{Beschreibung}
%   \label{fig:}
% \end{SCfigure}
\usepackage{sidecap}

% Befehl für "Entspricht"-Zeichen
\newcommand{\corresponds}{\ensuremath{\mathrel{\widehat{=}}}}

%Fußnoten zwingend auf diese Seite setzen
\interfootnotelinepenalty=1000

%Für chemische Formeln (von www.dante.de)
%% Anpassung an LaTeX(2e) von Bernd Raichle
\makeatletter
\DeclareRobustCommand{\chemical}[1]{%
  {\(\m@th
   \edef\resetfontdimens{\noexpand\)%
       \fontdimen16\textfont2=\the\fontdimen16\textfont2
       \fontdimen17\textfont2=\the\fontdimen17\textfont2\relax}%
   \fontdimen16\textfont2=2.7pt \fontdimen17\textfont2=2.7pt
   \mathrm{#1}%
   \resetfontdimens}}
\makeatother

%Si Einheiten
\usepackage{siunitx}

%c++ Code einbinden
\usepackage{listings}
\lstset{numbers=left, numberstyle=\tiny, numbersep=5pt}


\begin{document}

\begin{titlepage}
\centering
\textsc{\Large Anfängerpraktikum der Fakultät für
  Physik,\\[1.5ex] Universität Göttingen}

\vspace*{4.2cm}

\rule{\textwidth}{1pt}\\[0.5cm]
{\huge \bfseries
  Versuch Kapillarität und Viskosität\\[1.5ex]
  Protokoll}\\[0.5cm]
\rule{\textwidth}{1pt}

\vspace*{3cm}

\begin{Large}
\begin{tabular}{ll}
Praktikant: &  Michael Lohmann\\
 &  Felix Kurtz\\
% &  Kevin Lüdemann\\
% &  Skrollan Detzler\\
 E-Mail: & m.lohmann@stud.uni-goettingen.de\\
 &  felix.kurtz@stud.uni-goettingen.de\\
% &  kevin.luedemann@stud.uni-goettingen.de\\
% &  skrollan.detzler@stud.uni-goettingen.de\\
 Betreuer: & Martin Ochmann\\
 Versuchsdatum: & 26.05.2014\\
\end{tabular}
\end{Large}

\vspace*{0.8cm}

\begin{Large}
\fbox{
  \begin{minipage}[t][2.5cm][t]{6cm} 
    Testat:
  \end{minipage}
}
\end{Large}

\end{titlepage}

\tableofcontents

\newpage

\section{Einleitung}
\label{sec:einleitung}
In diesem Versuch sollen folgende Effekte untersucht werden:
Kapillarität, die bei der Wechselwirkung zwischen Flüssigkeiten und Oberflächen auftritt. Viskosität, die innere Reibung eines Fluids.\\
Diese beiden Phänomene spielen im alltäglichen Leben eine große Rolle und sind deshalb besonders untersuchenswert.
Pflanzen transportieren Wasser und darin enthaltene Nährstoffe durch Kapillarwirkung von den Wurzeln nach oben in die Blätter.\\
Die Viskosität einer Flüssigkeit beeinflusst maßgeblich deren Fließverhalten.
So ist der Asphalt der Straßen nicht fest, sondern nur sehr zähflüssig und kann so (in kleinem Maße) Spannungen und Risse ausgleichen.

\section{Theorie}
\label{sec:theorie}
\subsection{Kapillarität}
Zwischen Molekülen wirken \textit{Dipolkräfte} und \textit{Van-der-Waals-Kräfte}.
Van-der-Waals-Kräfte bilden sich vor allem bei langkettigen Molekülen aus, die polarisierbar sind.
Wird dann durch die statistische Bewegung der Elektronen für eine kurze Zeit ein Dipol ausgebildet, kann dieser per Influenz ein nahegelegenes Molekül ebenfalls polarisieren und zieht dieses an.
\begin{figure}[!h]
 \centering
 \includegraphics[width=0.5\linewidth]{Oberflaechenspannung.png}
 \caption{Modell zur Oberflächenspannung aus \cite[S. 198]{gerthsen} }
\end{figure}
An der Oberfläche einer Flüssigkeit wirken also nur die Kräfte ins Innere der Flüssigkeit, während in der Flüssigkeit auf ein Molekül Kräfte von allen Seiten wirken und sich so ausgleichen.\\
Die Oberflächenspannung einer Flüssigkeit $\sigma$ ist durch den Energiegewinn $dW$ definiert, der sich ergibt wenn sie eine Oberfläche $dA$ benetzt: $ \sigma=\frac{dW}{dA}$\\
Ist eine Kapillare mit Radius $r$ und eine Flüssigkeit mit der Oberflächenspannung $\sigma$ und der Dichte $\rho$ gegeben, so wird diese in der Kapillare um $h$ ansteigen, da sie durch das Benetzen Energie gewinnt.
Diese wird in potentielle Energie umgewandelt.
Es gilt also:
$$ \sigma\cdot A=mgh$$
Dabei ist $A=2\pi~r \cdot h$ und $m=\rho \pi ~ r^2 \cdot h$.
Für die Steighöhe $h$ folgt also:
\begin{align}
	h = \frac{2\sigma}{\rho~g~r}
\end{align}
oder für die Oberflächenspannung $\sigma$
\begin{align}
 \sigma =\frac{1}{2}h~\rho~g~r \label{eq:oberfl}
\end{align}
Man unterscheidet zwischen \textit{Kohäsion}, Kräfte zwischen gleichartigen Molekülen, und \textit{Adhäsion}, Kräfte, die an Grenzschichten auftreten.\\


\subsubsection{Mohrsche Waage}
Mit der Mohrschen Waage kann man die Dichte einer Flüssigkeit bestimmen.
Sie beruht auf dem \textit{archimedische Prinzip}, welches besagt, dass die Auftriebskraft eines Körpers so groß ist, wie die Gewichtskraft der verdrängten Flüssigkeit.\\
\begin{figure}[!htb]
	\centering
	\includegraphics[scale=0.7]{MohrscheWaage.png}
	\caption{Mohrsche Waage \cite{lp}}
	\label{fig:MohrscheWaage}
\end{figure}
Zuerst wird die Waage außerhalb der Flüssigkeit so eingestellt, dass sie sich in der Gleichgewichtslage befindet.
Dann taucht man den Probekörper in die Flüssigkeit, deren Dichte bestimmt werden soll.
Aufgrund der Auftriebskraft beginnt der Körper zu steigen.
Deshalb hängt man an die Hebelarmseite des Probekörpers kleine Gewichte verschiedener Massen und in unterschiedlichem Abstand zum Drehpunkt, sodass die Gleichgewichtslage wiederhergestellt wird.
Das von diesen Gewichten verursachte Drehmoment entspricht dem Drehmoment der Auftriebskraft.
Wenn man das Volumen des Probekörpers nicht kennt, muss zuerst Wasser als Referenz genommen werden.\\
\begin{align}
\rho_F=\frac{\sum\limits_{i=1}^nm_{F,i}\cdot r_{F,i}}{\sum\limits_{i=1}^nm_{W,i}\cdot r_{W,i}}\cdot\rho_W
\end{align}
Dabei ist $m_{F,i}$ die i-te Masse der Flüssigkeit F, welche im Abstand $r_{F,i}$ angehängt wurde.
$\rho_W$ bezeichnet hierbei die Dichte von Wasser, die   $\rho_W\approx 997~\si{\kilo\gram/\meter^3}$ beträgt \cite[S. 258]{gerthsen}.
\subsection{Viskosität}
Viskosität ist ein Maß dafür, wie zähflüssig eine Flüssigkeit ist, also wie stark die innere Reibung ist.\\
Man unterscheidet zwei Fälle von Strömungen, wenn ein Körper von einem anderen Medium  umflossen wird:
\begin{itemize}
	\item \textit{laminare} Strömung: Das Fluid strömt in Schichten, die sich nicht miteinander vermischen. %wikipedia
	\item \textit{turbulente} Strömung: Es kommt zu Verwirbelungen. Die Beschreibung dieser ist sehr komplex, soll hier aber nicht näher betrachtet werden.
\begin{figure}[htb]
  \centering
  \subfigure[Laminare Strömung\label{fig:laminar}]
  {\includegraphics[width=0.50\textwidth]{laminar}}
  \hfill
  \subfigure[Turbulente Strömung\label{fig:turbulent}]
  {\includegraphics[width=0.45\textwidth]{turbulent}}
  \caption{Verschiedene Strömungsarten aus \cite[S. 465]{giancoli}}
  \label{fig:label-gesamt}
\end{figure}
\end{itemize}
Die Reynoldszahl $Re$ ist ein Maß für den Übergang von laminarer zu turbulenter Strömung.
Sie ist so definiert:
\begin{align}
	Re=\frac{\rho~v~d}{\eta}
\end{align}
mit der Dichte $\rho$, der Fließgeschwindigkeit $v$, der charakteristischen Länge $d$ des umflossenen Gegenstandes und der Viskosität $\eta$ der Flüssigkeit.\\
Diese wird durch die folgende Bewegungsgleichung von Flüssigkeiten unter innerer Reibung definiert:
\begin{align}
	F=\eta~ A\frac{dv}{dr}
\end{align}
Setzt man diese Kraft gleich der Druckkraft $F_p=\pi r^2 \cdot (p_1-p_2)$ kann das \textit{Hagen-Poiseuille Gesetz} der laminaren Rohrströmung hergeleitet werden, welches den Volumenstrom $\dot V$ durch ein Rohr der Länge $l$ und des Radiuses $r$ beschreibt.\cite[S.125]{gerthsen} % hier muss noch ergänzt werden
\begin{align}
	\dot{V}=\frac{\pi(p_1-p_2)}{8\eta l}r^4
\end{align}

\section{Durchführung}
\label{sec:durchfuehrung}
Zuerst werden die Durchmesser der 3 verschiedenen Kapillaren jeweils dreimal mit dem Messmikroskop vermessen.
Dabei wird die Kapillare fokussiert, dann der linke oder rechte Rand anvisiert.
Die Stellung der Micrometerschraube wird abgelesen, bevor man die gegenüberliegende Seite anvisiert und erneut die Skala abliest.
Aus der Differenz der beiden ergibt sich der Durchmesser der Kapillare.\\
Bei den Versuchen wird Methanol verwendet, welches giftig ist.
Auch Ethylenglykol ist gesundheitsschädlich.
Deshalb ist bei allen Versuchen der Kontakt mit diesen zu vermeiden.

\subsection{Kapillarität}
Man reinigt die Kapillaren gründlich mit Lösungsmittel und destilliertem Wasser, bevor man sie mit der Wasserstrahlpumpe trocknet.
Dieser Vorgang muss später nach jeder Flüssigkeit wiederholt werden.
Dabei ist darauf zu achten, dass beim Trocknen alle Flüssigkeitsreste entfernt werden und sich ganz besonders kein Film am oberen Ende bildet, der einen erheblich größeren Wiederstand beim Steigen hervorrufen würde.\\
Nun füllt man sich die drei auf Oberflächenspannung zu untersuchende Flüssigkeiten destilliertes Wasser, Methanol und Ethylenglykol in einen Becher ab.
Mithilfe der Mohr'schen Waage (Abb. \ref{fig:MohrscheWaage}) bestimmt man die jeweilige Dichte.
Dabei ist darauf zu achten, dass der Probekörper sauber ist und ganz in die Flüssigkeit eintaucht.\\
Dann misst man für jede der drei Flüssigkeiten und für jede der drei Kapillaren jeweils dreimal den Höhenunterschied $h_{Kap}$ der Flüssigkeitspegel, der sich ergibt, wenn man die Kapillare in die Flüssigkeit taucht und anschließend bis zur Oberfläche herauszieht.\\
Anschließend werden Methanol und Ethylenglykol in spezielle Behälter gegossen und somit vorschriftsmäßig entsorgt.
Nun müssen alle Gefäße und die Kapillaren gereinigt werden.
\begin{figure}[!htb]
	\centering
	\includegraphics[scale=0.6]{ViskoAufbau.png}
	\caption{Versuchsaufbau zur Messung der Viskosität \cite{lp}}
	\label{fig:Visko}
\end{figure}
\subsection{Innere Reibung}
Zuerst misst man den Durchmesser des Glaszylinders und den Abstand der Strichmarken 50 und 45 von diesem, die Länge der Kapillaren und die Temperatur des destillierten Wassers.
Dann befestigt man eine der Kapillaren am Auslaufstutzen des Zylinders, hält die Öffnung zu und befüllt alles mit destillierten Wasser bis zur Strichmarke 50.
Es wird die Ausflusszeit bis zum Erreichen der Strichmarke 45 gemessen.
Dabei ist das ausfließende Wasser in einem Gefäß aufzufangen und wiederzuverwenden.
Diesen Vorgang wiederholt man auch für die anderen beiden Kapillaren.\\
Nun wählt man die Kapillare mit dem kleinsten Durchmesser und misst zu dieser während des Ausflusses die Zeit in Abhängigkeit der Höhe der Wassersäule.\\
Abschließend muss alles gesäubert werden und die Arbeitsfläche trocken gewischt werden.

\section{Auswertung}
\label{sec:auswertung}
\subsection{Dichte der Flüssigkeiten}
Um die Dichte von Methylalkohol und Ethylenglykol zu bestimmen, wurde die Mohr'sche Waage verwendet.
\begin{table}
\centering
\begin{tabular}{|c|c|c|c|c|}
\hline Aufhängung einzelner Gewichte & 5000mg & 500mg & 50mg &Dichte [kg/$m^3$]\\
\hline Dest. Wasser  & 10	&	& 1	& 997	\\
\hline Äthylenglykol & 10	& 9	& 3	& 1088	\\
\hline Methylalkohol & 8	& 4	& 	& 837	\\\hline
\end{tabular}
\caption{Position der einzelnen Gewichte an der Moor'schen Waage in Skalenteilen bei den unterschiedlichen Flüssigkeiten\label{tab:dichte}}
\end{table}
Die Literaturwerte nach \cite{Formelsammlung} (S. 130-131) lauten für Ethylenglykol 1113kg/m$^3$ und für Methanol 790kg/m$^3$.
\subsection{Oberflächenspannung}
\begin{table}[!h]
\centering
\begin{tabular}{|l|l|c|c|}
\hline
Flüssigkeit 		&Kapillar & m. Steighöhe [cm]	& Oberflächenspannung [$10^{-3}$N/m]\\\hline\hline
			&grün	& $1.45\pm 0.04$		&\\
Destiliertes Wasser	&blau	& $2.327\pm 0.035$		&\\
                        &braun  & $3.23\pm 0.04$		&\\
\hline
			&grün	& $0.85\pm 0.04$		&\\
Ethylenglykol		&blau	& $1.383\pm 0.022$		&\\
			&braun	& $1.98\pm 0.06$		&\\
\hline
			&grün	& $0.517\pm 0.022$		&\\
Methylalkohol		&blau	& $0.98\pm 0.10$		&\\
			&braun	& $1.33\pm 0.04$		&\\
\hline
\end{tabular}
\caption{Steighöhe unterschiedlicher Flüssigkeiten in unterschiedlichen Kapillaren}
\end{table}
Daraus lässt sich die Oberflächenspannung der drei Flüssigkeiten bestimmen.
Sie berechnet sich aus der Formel
\begin{align}
\sigma=\frac{1}{2}h\rho\cdot g r
\end{align} 
Wobei $h$ die Steighöhe im Kapillar mit dem Radius $r$ der Flüssigkeit mit der Dichte $\rho$ ist und g die Erdbeschleunigung.
\section{Diskussion}
\label{sec:diskussion}
Aufgrund von fehlender Zeit schafften wir es leider nicht, die Messung der Ausflusszeit des mittleren Kapillars zu bestimmen.
Da wir jedoch die Messungen des kleinen und großen Kapillars durchführen konnten, haben wir wenigstens einen Eindruck, wie die Kapillardicke mit der Ausflussgeschwindigkeit zusammenhängt.

\section{Anhang}
\begin{thebibliography}{9}

\bibitem{lp} 
	\emph{Lehrportal der Universität Göttingen, Kapillarität und Viskosität},
  https://lp.uni-goettingen.de/get/text/3638, abgerufen 22.06.14 18:32 Uhr

\bibitem{gerthsen}
	\textsc{Dieter Meschede} (2010): \emph{Gerthsen Physik}, 24. Auflage, Springer Heidelberg
Dordrecht London New York

\bibitem{Formelsammlung}
	\textsc{Wolfgang Pfeil et. al.} (2009): \emph{Das große Tafelwerk (interaktiv)}, 1. Auflage, Cornelsen Berlin

\bibitem{giancoli}
	\textsc{Douglas C. Giancoli} (2010): \emph{Physik - Lehr- und Übungsbuch}, 3. Auflage, Pearson Studium London
\end{thebibliography}


\end{document}
